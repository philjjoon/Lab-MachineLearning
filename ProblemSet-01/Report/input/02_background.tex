\chapter{Application}
\label{application}
%#############################################################################################

In this chapter, we are trying to apply the algorithms that are described and implemented in~\ref{implementation} to various datasets: \textit{usps}, \textit{banana}, \textit{fishbowl}, \textit{swissroll} and \textit{flatroll}.

\section{Assignment 4: PCA}
\label{assignment4}

This assignment askes us to apply \textit{PCA} to the \textit{usps} data set and visualizing the results. The \textit{usps} dataset consists of images of digit zero to nine, and each digit can be viewed as a class. Firstly, i seperate the data set according to each digit into ten classses and then applied \textit{PCA} to each class. The \textit{PCA} was applied to the original data set and noisy data set.

\subsection{Original Data Set}
\label{ass4:original}

The original data set contains of 2007 images and each image has a dimension of \textit{16 x 16}. 

\subsection{Noisy Data Set}
\label{ass4:noisy}



%#############################################################################################

\section{Assignment 5: Outlier Detection Using $\gamma$-Index}
\label{assignment5}

%#############################################################################################

\section{Assignment 6: LLE}
\label{assignment6}

%#############################################################################################

\section{Assignment 7: LLE With Noise}
\label{assignment7}