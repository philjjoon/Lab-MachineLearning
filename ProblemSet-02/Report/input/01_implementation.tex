\chapter{Implementation}
\label{chap:implementation}
%#############################################################################################

This chapter describes the implementation part of the second problem set. There are five assignments in this part: 1) implement \textit{k-means} clustering, 2) implement stepwise optimal \textit{hierarchical agglomerative} clustering, 3) implement a function which given a hierarchical clustering sets up a \textit{dendogram} plot, 4) implement the \textit{EM} algorithm for \textit{Gaussian Mixture Models (GMM)} and 5) implement a function that visualizes the \textit{GMM} for two-dimensional data.

\section{Assignment 1: K-Means Clustering}
\label{sec:assignment1}

In this assignment the \textit{k-means} clustering algorithm should be implemented as follows:

\begin{center}
\texttt{mu, r = kmeans(X, k, max\_iter=100)}
\end{center}

The algorithm terminates when the membership and the cluster centers no longer change or after \textit{max\_iter} (default value = 100) iteration, depending on which comes first. The implemented function was tested on the test data and passed the test.

%######################################################################################
\section{Assignment 2: Hierarchichal Agglomerative Clustering}
\label{sec:assignment2}

The task in this assignment is to implement stepwise optimal \textit{hierarchichal agglomerative} clustering with \textit{k-means} criterion as a function. The implemented function was tested on the test data and passed the test.

\begin{center}
\texttt{R, kmloss, mergeidx = kmeans\_agglo(X, r)}
\end{center}

%######################################################################################
\section{Assignment 3: Dendrogram Plot}
\label{sec:assignment3}

The third assignment in the implementation part is to implement a function which given a hierarchical clustering sets up a \textit{dendogram} plot:

\begin{center}
\texttt{agglo\_dendro(kmloss, mergeidx)}
\end{center}

The parameters \textit{kmloss} and \textit{mergeidx} correspond to the results of \textit{kmeans\_agglo}. The function \textit{scipy.cluster.hierarchy.dendogram} is used to draw the \textit{dendogram} plot.

%######################################################################################
\section{Assignment 4: Expectation-Maximization}
\label{sec:assignment4}

In this assignment the \textit{EM} algorithm for \textit{Gaussian Mixture Model (GMM)} should be implemented as a function:

\begin{center}
\texttt{pi, mu, sigma = em\_gmm(X, k, max\_iter=100, init\_kmeans=False)}
\end{center}

The parameter \textit{init\_kmeans} determines the initialization method. If it is true, then \textit{k-means} is used for the initialization. For random initialization, \textit{k} data points are selected as cluster centers, the prior \textit{pi} of each cluster is set to $1 / k$. The sigma of each cluster is the identity matrix. On the other hand, the cluster centers of \textit{k-means} are used for \textit{k-means} initialization. The prior \textit{pi} of each cluster is set to total data points in each cluster divided by total number of data points in the data set. The sigma of each cluster is the covariance matrix of the data points of each cluster.

% TODO: regularization of covariance matrix

The algorithm terminates when the maximal number of iterations \textit{max\_iter} has been reached or the log likelihood no longer changes ($<0.001$)


%######################################################################################
\section{Assignment 5: EM Visualization}
\label{sec:assignment5}

In the last assignment a function to visualize the \textit{GMM} should be implemented. The figure should show the data as a scatter plot, the mean vectors as red crosses and the covariance matrix as ellipses (centered at the mean).